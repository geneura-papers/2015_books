\documentclass[preprint]{elsarticle}
\biboptions{round, numbers}
\usepackage[latin1]{inputenc}
%\usepackage[T1]{fontenc}
%\usepackage{textcomp}
\usepackage{graphicx}
\usepackage{color}
%\usepackage{setspace}
\usepackage{url}
\usepackage[english]{babel}

\begin{document}

%%%%%%%%%%%%%%%%%%%%%%%%%%%%%%%   TITLE   %%%%%%%%%%%%%%%%%%%%%%%%%%%%%%%

\title{Applying Computational Intelligence Methods for Predicting the Sales of Newly Published Books in a Real Editorial Business Management Environment: Response to Reviewers\' comments}

\noindent
Dear Sirs,\\

First, we would like to express our gratitude to the Editor in Chief,
the editorial team and the reviewers whose valuable comments on this
paper have significantly improved its quality. All suggestions are
highly appreciated and we have added a line in the acknowledgements
section thanking the reviewers. \\ 

In this new revision, we have carefully addressed all the comments in
the new version of the paper. We will enumerate below these comments
(in italics) and our corresponding response. \\ 

We look forward to hearing from you about the final decision on our paper. \\

\noindent
Yours sincerely,\\
The authors.


%------------------------------------------------------------------------------------
\section{Comments by Reviewer \#1}

\noindent \emph{There are some issues needed to be addressed further:} \\

\noindent (1) \emph{Generally speaking, predicting sales is regarded as a regression probelm (numerical output) rather than classification (categorical one). Both CFS and RelieFF are originally designed for classification problem. When CFS is used for regression, please to introduce how to estimate correlations between each feature and output (total sales). Since RReliefF, as a variant of ReliefF, is special for regression, why do not the authors use RReliefF? } 

\begin{verbatim}
In all our experiments we have applied the Weka's implementation 
of the ReliefF algorithm in the regression problems. 
This version is called "Regressional ReliefF" or "RReliefF" for 
short, which was first introduced and implemented by Robnik-Sikonja 
and Kononenko (in ref. [73] for numeric targets). 
Therefore, we have clarified this point in the text to remove 
any ambiguity.

For CSF, estimating the correlation between attributes is 
standard linear (Pearson's) correlation. The details have 
been added to the text to clarify this point.
\end{verbatim}

~\\
\noindent (2) \emph{In sub-section 4.2, five methods are introduced, i.e.,M5P,kNN,RF,LR and SVM. In sub-section 5.1, why is MLP (multi-layer perceptron) with one hidden layer described ? For RF, how many trees are set? } 

\begin{verbatim}
In sub-section 5.1 the MLP was described as our initial intention 
was to include it in this study. Thus, in sub-section 
''2.1. Selecting an appropriate forecasting method'' (old 
sub-section 4.2) six methods have been introduced M5P, kNN, 
RF, LR, SVM and MLP.

Also, as suggested by the second reviewer (see his first comment), 
the results obtained with this method have been included in the 
new version of this paper.

As far as the number of trees is related, in the experiments 
section, when describing the parameter settings, it has been 
mentioned that the best performance was obtained with number of 
trees equals to 50.
\end{verbatim}

~\\
\noindent (3) \emph{As we known, tree-based mehods (e.g.,M5P and RF) themselve embed feature selection procedure. That is, some features are choosen to build trees. It is not fair enough to compare five methods in Table 3. } 

\begin{verbatim}
Although we agree with the reviewer's comment, and the RF and M5P 
have an embedded feature selection mechanism, we believe that it 
is unfair to compare them with the filter based feature selection 
methods we have deployed in our research. 
The idea is simply to show the decision maker what are the possible 
results when we apply one of the off-the-shelf classifiers available 
in a common open source framework such as WEKA. 
Moreover, it is not uncommon in the literature to see such comparison 
between classifiers, such as in the paper:
  M. Fernandez-Delgado, E. Cernadas, S. Barro, and D. Amorim. 2014. 
  Do we need hundreds of classifiers to solve real world 
  classification problems?. J. Mach. Learn. Res. 15, 1, pp.3133-3181
which incorporates a general comparison between all these classifiers.
\end{verbatim}

~\\
\noindent (4) \emph{It seems more reasonable to organize and write that paper according to regression viewpoint only. } 

\begin{verbatim}
According to the reviewer's comment, we have addressed this issue
at the same time that comment #1, where the application procedure 
of feature selection methods for regression problems has been 
clarified. Also, we have added the ''regression'' term all along 
the paper including the description of the dataset.
\end{verbatim}

~\\
\noindent (5) \emph{It is necessary to write that paper more concisely. } 

\begin{verbatim}
Following your suggestion we have reorganized some parts of the 
text and several paragraphs have been rewritten in order to make 
them shorter, clearer and to improve its comprehension. 

Also, several tables have been moved to the Appendix and that
data has been presented now as the new Figure 3.

Finally, the paper has been completely revised and several 
typographic errors have been fixed.
\end{verbatim}



%------------------------------------------------------------------------------------
\section{Comments by Reviewer \#2}

\noindent \emph{This paper applies some machine learning technologies to address issues related to book sales. There are some revisions that should be addressed by authors: } \\


\noindent (1) \emph{The ANN and ELM (extreme learning machine) should be added into the experiments; } 

\begin{verbatim}
We have added the results of ANN (MLP) and ELM to all tables 
accordingly. A description of both approaches have been also 
added to subsection 2.1.
\end{verbatim}

~\\
\noindent (2) \emph{The authors should clarify their main contributions in the introduction parts.}

\begin{verbatim}
We have rewritten the introduction and introduced this new text:

Our main contribution is, thus, to develop a methodology to 
process book sales data in order to first find out relevant 
variables and then use these variables to process data so that 
the publisher, using them, can estimate sales and from them order 
a specific print run, instead of standard ones, which are around 
300 copies for standard books and around 5000 for those that are 
considered, in advance, best sellers.

Also, in the next-to-last paragraph, we have laid out our 
methodology and the main contributions of this paper.
\end{verbatim}

~\\
\noindent (3) \emph{Technologies related to recommend systems may have the ability to deal with this problem. The authors should add the recommend systems into the future work part. } 

\begin{verbatim}
Taking into account the reviewer's comment, the recommendation 
systems have been added to the future works section, at the end 
of Section 7 (see page 19).
\end{verbatim}

~\\
\noindent (4) \emph{Some tables should be transfer to figures because there are too many tables. } 

\begin{verbatim}
Following your suggestion several tables have been moved to 
the Appendix and that data has been summarized and presented 
as the new Figure 3 in order to make that numeric data clearer 
and to improve its comprehension.
\end{verbatim}

~\\
\noindent (5) \emph{Graphical abstract should be added. This paper is organized well. The authors should revise their work according to above comments. } 

\begin{verbatim}
Taking into account the reviewer's comment, we have designed 
a graphical abstract as a visual summary of the main findings 
of the article.

We would like to thank this reviewer for his suggestions and 
advice about our paper.
\end{verbatim}



%------------------------------------------------------------------------------------
\section{Comments by Reviewer \#3}

\noindent \emph{There are several concerns should be considered.}\\

\noindent (1) \emph{In section introduction, this manuscript should describe the contribution of this work. } 

\begin{verbatim}
We have rewritten the introduction and introduced this new text:

Our main contribution is, thus, to develop a methodology to 
process book sales data in order to first find out relevant 
variables and then use these variables to process data so that 
the publisher, using them, can estimate sales and from them order 
a specific print run, instead of standard ones, which are around 
300 copies for standard books and around 5000 for those that are 
considered, in advance, best sellers.

Also, in the next-to-last paragraph, we have laid out our 
methodology and the main contributions of this paper.
\end{verbatim}

~\\
\noindent (2) \emph{sections 4.1 and 4.2 should be moved to section 2. } 

\begin{verbatim}
Following your suggestion we have reorganized these sub-sections, 
moving them to section 2.
\end{verbatim}

~\\
\noindent (3) \emph{In section 4, this manuscript should show a flowchart of the proposed forecasting scheme and describe it. } 

\begin{verbatim}
Taking into account the reviewer's comment, we have improved the 
description of the methodology by including a flowchart (see Figure 1 
at page 6).
\end{verbatim}

~\\
\noindent (4) \emph{In Section 5, parameter settings of M5P, kNN and RF also should be shown. } 

\begin{verbatim}
Taking into account the reviewer's comment, we have improved the 
section 5, including the parameter settings of M5P, kNN and RF 
methods.
\end{verbatim}


\end{document}
