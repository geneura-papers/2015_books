\documentclass[preprint]{elsarticle}
\biboptions{round, numbers}
\usepackage[latin1]{inputenc}
%\usepackage[T1]{fontenc}
%\usepackage{textcomp}
\usepackage{graphicx}
\usepackage{color}
%\usepackage{setspace}
\usepackage{url}
\usepackage[english]{babel}

\begin{document}

%%%%%%%%%%%%%%%%%%%%%%%%%%%%%%%   TITLE   %%%%%%%%%%%%%%%%%%%%%%%%%%%%%%%

\title{Applying Computational Intelligence Methods for Predicting the Sales of Newly Published Books in a Real Editorial Business Management Environment: Response to Reviewers\' comments}

\noindent
Dear Sirs,\\

First, we would like to express our gratitude to the Editor in Chief,
the editorial team and the reviewers whose valuable comments on this
paper have significantly improved its quality. All suggestions are
highly appreciated and we have added a line in the acknowledgements
section thanking the reviewers. \\ 

In this new revision, we have carefully addressed all the comments in
the new version of the paper. We will enumerate below these comments
(in italics) and our corresponding response. \\ 

We look forward to hearing from you about the final decision on our paper. \\

\noindent
Yours sincerely,\\
The authors.


%------------------------------------------------------------------------------------
\section{Comments by Reviewer \#1}

\noindent \emph{There are some issues needed to be addressed further:} \\

\noindent (1) \emph{Generally speaking, predicting sales is regarded as a regression problem (numerical output) rather than classification (categorical one). Both CFS and RelieFF are originally designed for classification problem. When CFS is used for regression, please to introduce how to estimate correlations between each feature and output (total sales). Since RReliefF, as a variant of ReliefF, is special for regression, why do not the authors use RReliefF? } 

\begin{verbatim}
In all our experiments we have applied  WEKA's implementation 
of the ReliefF algorithm in the regression problems. 
This version is called "Regressional ReliefF" or "RReliefF" for 
short, which was first introduced and implemented by Robnik-Sikonja 
and Kononenko (in ref. [73] for numeric targets). 
Therefore, we have clarified this point in the text to remove 
any ambiguity.
\end{verbatim}
% Antonio - TODO: say where is this clarification in the text (page/paragraph) 
\begin{verbatim}
For CSF, estimating the correlation between attributes is 
standard linear (Pearson's) correlation. The details have 
been added to the text to clarify this point.
\end{verbatim}
% Antonio - TODO: say where is this clarification in the text (page/paragraph) 


~\\
\noindent (2) \emph{In sub-section 4.2, five methods are introduced, i.e.,M5P,kNN,RF,LR and SVM. In sub-section 5.1, why is MLP (multi-layer perceptron) with one hidden layer described ?} 

\begin{verbatim}
The current paper includes in sub-section 
`2.1. Selecting an appropriate forecasting method' (former
sub-section 4.2) six methods: M5P, kNN, RF, LR, SVM, MLP,
 and ELM (requested by Reviewer #2).

Moreover, as suggested by the Reviewer #2 (see his/her first comment), 
the results obtained with the two last methods have been included in 
this new version of the paper.
\end{verbatim}
% Antonio - TODO: say where are these results (table/figure/page)
% [pedro] TODO => decir que se han añadido ciertas refs sobre ELM

~\\
\noindent (2) \emph{For RF, how many trees are set? }

\begin{verbatim}
As far as the number of trees is concerned, in the experiments 
section, when describing the parameter settings, 
it has been mentioned that the best performance was obtained with a 
number of trees equal to 50.
\end{verbatim}
% Antonio - TODO: say where is this clarification in the text (page/paragraph) 


~\\
\noindent (3) \emph{As we known, tree-based methods (e.g., M5P and RF) themselves embed feature selection procedure. That is, some features are chosen to build trees. It is not fair enough to compare five methods in Table 3. } 

\begin{verbatim}
Although we agree with the reviewer's comment, and the RF and M5P 
have an embedded feature selection mechanism, we believe that it 
is unfair to compare them with the filter based feature selection 
methods we have deployed in our research. 
The aim is simply to show to the decision maker what are the possible 
results when we apply one of the off-the-shelf classifiers available 
in a common open source framework such as WEKA. 
Moreover, it is not uncommon in the literature to see such comparison 
between classifiers, as in the paper:
  M. Fernandez-Delgado, E. Cernadas, S. Barro, and D. Amorim. 2014. 
  Do we need hundreds of classifiers to solve real world 
  classification problems?. J. Mach. Learn. Res. 15, 1, pp.3133-3181
which incorporates a general comparison between all these classifiers.
\end{verbatim}
%Every response should end with changes made in the paper, even if it
%is only clarification. It is not an invitation to discussion. 
% Antonio - TODO: say how this has been addressed in the paper (where it is clarified). Say if this reference has been included as new.
% JJ - Antonio, TODO by whom? If it's an issue, put it in the repo and assign it or do it yourself.
~\\
\noindent (4) \emph{It seems more reasonable to organize and write that paper according to regression viewpoint only. } 

\begin{verbatim}
The reviewer is right, thus, we have revised the whole paper remarking 
the regression topic present on the work. We have clarified in several
 paragraphs that the problem has been addressed by means of regression 
methods (for instance, note the new title of Section 2.1), or we have 
generally referred to the term `regression' instead of `classification', 
in order to better suit the regression viewpoint.
Moreover this suggestion have been also addressed solving the issue 
related to comment #1, where the application procedure of feature 
selection methods for regression problems has been clarified. 
\end{verbatim}

~\\
\noindent (5) \emph{It is necessary to write that paper more concisely. } 

\begin{verbatim}
Following reviewer's suggestion we have reorganized some parts 
of the text, removed some sentences, and rewritten several paragraphs 
in order to make them shorter, clearer and to improve its comprehension. 

Also, all the tables of results have been moved to the Appendix and 
that data has been presented now as the new Figures XX, YY, ZZ.

Finally, the paper has been completely revised and several 
typographic errors have been fixed. 

We are very grateful the reviewer's valuable suggestions and comments. 
\end{verbatim}
% Antonio - TODO: show some examples of reduced text (page/paragraph), there will be some new figures, I think.
% we have to say: "Paragraph #X in subsection #Y" (naming the paragraphs that we identified and gathered-joined)


~\\

%------------------------------------------------------------------------------------
\section{Comments by Reviewer \#2}

\noindent \emph{This paper applies some machine learning technologies to address issues related to book sales. There are some revisions that should be addressed by authors: } \\


\noindent (1) \emph{The ANN and ELM (extreme learning machine) should be added into the experiments; } 

\begin{verbatim}
Following the reviewer's request, we have conducted several new experiments 
and added the results of ANN (MLP) and ELM to Tables XX, YY, ZZ and
 Figures AA, BB accordingly. 
A description of both approaches have been also included in subsection 2.1.

In addition new references has been added to the paper related to these 
methods, namely [REF_XX, REF_YY].
\end{verbatim}
% Antonio - TODO: Complete with the number of tables and figures, along with the new references
% I don't know what references have been added in this revision. I'll ask Hossam.


~\\
\noindent (2) \emph{The authors should clarify their main contributions in the introduction parts.}

\begin{verbatim}
We have rewritten the introduction and included new paragraphs to better 
explain the problem and the aim of the paper. We have also remarked the
 contribution with the paragraphs:

"Our intention in this paper is to find out the main factors influencing
sales in order to create a tool that the publisher can use to decide how
many books should be printed, as well as how to leverage these printed
copies to maximize sales using the decision variables under his control. 
That is why, using data obtained from a company that sells software for 
publishers, we analyse them and compute predictive models that can be 
mainly used as decision-aid tools for book publishers. With these models,
 publishers will be able to combine their expert knowledge about the market 
with the created forecasting models in order to get a reliable estimation of
 book sales, and, based on it, act consequently in order to maximize the 
books sold for a particular print run. This will improve the current 
decision flow that the publishers follow, which consist in analysing (in a
 subjective way) the quality of the book and its features (author, genre, 
etc), and take a `fuzzy' decision roughly between printing 300 copies 
(standard book) or 5000 (best seller).

Our main contribution is, thus, the development of the methodology 
to process book sales data in order to, first find out relevant variables, 
and then, use these variables to process the data to accurately estimate 
future sales. To our knowledge this is the first work in which regression
 methods have been applied to estimate sales for new launched books. 
Moreover, the use of a real dataset of book sales is another point to 
remark with respect to other works."

Also, in the following paragraphs, we have laid out our 
methodology to differentiate it from other approaches.

\end{verbatim}
% Antonio - TODO: I rewritten this text, so we should copy the new one here. But I think we all should contribute to create a very good justification (see next CONCERN). :D
% Antonio - CONCERN: We should definitely write a convincing text which separates this work from existing approaches. Maybe the use of real data is a good point. And also emphasize the contribution to the SotA.
% [Pedro] well, no previous work on "print run & book sales prediction" has been done (at least, I have not found anything in the bibliography)
% [Hossam] about the contribution, I agree with you. We can mention that we used real data in our work.


~\\
\noindent (3) \emph{Technologies related to recommend systems may have the ability to deal with this problem. The authors should add the recommend systems into the future work part. } 

\begin{verbatim}
Taking into account the reviewer's comment, the recommendation 
systems have been added to the future work section, at the end 
of Section 7 (see page XX).
\end{verbatim}
% Antonio - TODO: put the correct page when the text is finished

~\\
\noindent (4) \emph{Some tables should be transfer to figures because there are too many tables. } 

\begin{verbatim}
Following reviewer's suggestion, all the results tables have been 
moved to the Appendix and that data has been summarized and 
presented as the new Figures XX, YY, ZZ in order to make that numeric 
data clearer and to improve their comprehension.
\end{verbatim}
% Antonio - TODO: put the correct page when the text is finished

~\\
\noindent (5) \emph{Graphical abstract should be added. This paper is organized well. The authors should revise their work according to above comments. } 

\begin{verbatim}
We agree with the reviewer and have designed (and uploaded) a 
graphical abstract as a visual summary of the paper structure 
and the utility of the obtained results.


We would like to thank this reviewer for his/her suggestions 
and comments to improve the quality of our paper.
\end{verbatim}

~\\


%------------------------------------------------------------------------------------
\section{Comments by Reviewer \#3}

\noindent \emph{There are several concerns should be considered.}\\

\noindent (1) \emph{In section introduction, this manuscript should describe the contribution of this work. } 

\begin{verbatim}

The reviewer is right. The main contribution of this paper to the 
state of the art was not properly described. Thus, as stated above 
in the second response to Reviewer #2, we have clarified this in 
the introduction adding new sentences and the paragraph:

"Our main contribution is, thus, the development of the methodology 
to process book sales data in order to, first find out relevant variables, 
and then, use these variables to process the data to accurately estimate 
future sales. To our knowledge this is the first work in which regression
 methods have been applied to estimate sales for new launched books. 
Moreover, the use of a real dataset of book sales is another point to 
remark with respect to other works."

\end{verbatim}
% Antonio - TODO - put here the paragraph as it will finally be (if it is changed again).

~\\
\noindent (2) \emph{sections 4.1 and 4.2 should be moved to section 2. } 

\begin{verbatim}
Following reviewer's suggestion we have reorganized these sub-sections, 
moving them to section 2.
\end{verbatim}


~\\
\noindent (3) \emph{In section 4, this manuscript should show a flowchart of the proposed forecasting scheme and describe it. } 

\begin{verbatim}
Taking into account the reviewer's comment, we have improved the 
description of the methodology including a flowchart (see Figure XX 
at page YY).
\end{verbatim}
% Antonio - CONCERN: I don't know if te reviewer is speaking about the whole methodology or just about the regression/forecasting part... The figure shows a rough summary of the methodology.
% Antonio - TODO: Improve the figure, which is a bit 'generic'. Put the correct figure and page.
% Almost every paper I have read (and that has been cited in the SotA) shows a very generic flowchat


~\\
\noindent (4) \emph{In Section 5, parameter settings of 
M5P, kNN and RF also should be shown. } 

\begin{verbatim}
Reviewer is right, so we have improved Section 5, including the 
parameter settings of M5P, kNN and RF methods (see paragraph XX at 
page YY), along with the configuration parameters for the rest of 
methods.

The authors want to express our gratitude to this reviewer 
for his/her suggestions.
\end{verbatim}
% Antonio - CONCERN: don't we also put the configuration of the rest of the methods? And say it. :D 
% Could we use a table for showing them easily?
% [pedro] TODO: design the table taken the parameters/values shown in section 5.1
% Antonio - TODO: say paragraph/page where they are.



% Antonio - TODO: Say if there are new references and include them in a final section in this letter.
% yes, there are some, but is Hossam who knows what references have been added in this revision.
%------------------------------------------------------------------------------------
\section{New references added after revision}

~\\
[53] G.B. Huang, Q.Y. Zhu, C.K. Siew, Extreme learning machine: theory and applications, Neurocomputing 70 (1) (2006) 489-501.

~\\
[54] G. Huang, G.B. Huang, S. Song, K. You, Trends in extreme learning machines: a review, Neural Networks 61 (2015) 32-48.

~\\
[60] Wikipedia, List of Dewey decimal classes, [Online; ac- cessed 05-May-2016]. https://en.wikipedia.org/wiki/List\_of\_Dewey\_Decimal\_classes

~\\
[71] M. A. Hall, Correlation-based feature selection for discrete and numeric class machine learning, in: Proceedings of the Seventeenth International Conference on Machine Learning, ICML2000, Morgan Kaufmann Publishers Inc., San Francisco, CA, USA, 2000, pp. 359-366.


\end{document}

