\documentclass[preprint]{elsarticle}
\biboptions{round, numbers}
\usepackage[latin1]{inputenc}
%\usepackage[T1]{fontenc}
%\usepackage{textcomp}
\usepackage{graphicx}
\usepackage{color}
%\usepackage{setspace}
\usepackage{url}
\usepackage[english]{babel}

\begin{document}

%%%%%%%%%%%%%%%%%%%%%%%%%%%%%%%   TITLE   %%%%%%%%%%%%%%%%%%%%%%%%%%%%%%%

\title{Applying Computational Intelligence Methods for Predicting the Sales of Newly Published Books in a Real Editorial Business Management Environment: Response to Reviewers\' comments}

\noindent
Dear Sirs,\\

First, we would like to express our gratitude to the Editor in Chief, the editorial team, and the reviewers whose valuable comments on this paper have significantly improved its quality. All suggestions are highly appreciated and we have added a line in the acknowledgements section thanking the reviewers. \\

In this new revision, we carefully have addressed all the comments in the new version of the paper. We will enumerate below these comments (in italics) and our correspondent answer to each of them. \\

We will look forward to hearing from you about the final decision on our paper. \\

\noindent
Yours sincerely,\\
The authors.


%------------------------------------------------------------------------------------
\section{Comments by Reviewer \#1}

\noindent \emph{There are some issues needed to be addressed further:} \\

\noindent (1) \emph{Generally speaking, predicting sales is regarded as a regression probelm (numerical output) rather than classification (categorical one). Both CFS and RelieFF are originally designed for classification problem. When CFS is used for regression, please to introduce how to estimate correlations between each feature and output (total sales). Since RReliefF, as a variant of ReliefF, is special for regression, why do not the authors use RReliefF? } 

\begin{verbatim}
TODO
\end{verbatim}

~\\
\noindent (2) \emph{In sub-section 4.2, five methods are introduced, i.e.,M5P,kNN,RF,LR and SVM. In sub-section 5.1, why is MLP (multi-layer perceptron) with one hidden layer described ? For RF, how many trees are set? } 

\begin{verbatim}
In sub-section 5.1 the MLP was described as our initial intention was 
to include it in this study. Thus, as suggested by the reviewer (see 
next comment), obtained results with this method have been included
in the new version of this paper.

As far as the number of trees is related, TODO.
\end{verbatim}

~\\
\noindent (3) \emph{As we known, tree-based mehods (e.g.,M5P and RF) themselve embed feature selection procedure. That is, some features are choosen to build trees. It is not fair enough to compare five methods in Table 3. } 

\begin{verbatim}
TODO
\end{verbatim}

~\\
\noindent (4) \emph{It seems more reasonable to organize and write that paper according to regression viewpoint only. } 

\begin{verbatim}
TODO
\end{verbatim}

~\\
\noindent (5) \emph{It is necessary to write that paper more concisely. } 

\begin{verbatim}
TODO
\end{verbatim}



%------------------------------------------------------------------------------------
\section{Comments by Reviewer \#2}

\noindent \emph{This paper applies some machine learning technologies to address issues related to book sales. There are some revisions that should be addressed by authors: } \\


\noindent (1) \emph{The ANN and ELM (extreme learning machine) should be added into the experiments; } 

\begin{verbatim}
TODO
\end{verbatim}

~\\
\noindent (2) \emph{The authors should clarify their main contributions in the introduction parts. } 

\begin{verbatim}
We have rewritten the introduction and introduced this new text:

Our main contribution is, thus, to develop a
methodology to process book sales data in order to first find out
relevant variables and then use these variables to process data so
that the publisher, using them, can estimate sales and from them order
a specific print run, instead of standard ones, which are around 300
copies for standard books and around 5000 for those that are
considered, in advance, {\em best sellers}.

Also, in the next-to-last paragraph, we have laid out our methodology 
and the main contributions of this paper.
\end{verbatim}

~\\
\noindent (3) \emph{Technologies related to recommend systems may have the ability to deal with this problem. The authors should add the recommend systems into the future work part. } 

\begin{verbatim}
TODO
\end{verbatim}

~\\
\noindent (4) \emph{Some tables should be transfer to figures because there are too many figures. } 

\begin{verbatim}
TODO
\end{verbatim}

~\\
\noindent (5) \emph{Graphical abstract should be added. This paper is organized well. The authors should revise their work according to above comments. } 

\begin{verbatim}
TODO
\end{verbatim}



%------------------------------------------------------------------------------------
\section{Comments by Reviewer \#3}

\noindent \emph{There are several concerns should be considered. } \\


\noindent (1) \emph{In section introduction, this manuscript should describe the contribution of this work. } 

\begin{verbatim}
We have rewritten the introduction and introduced this new text:

Our main contribution is, thus, to develop a
methodology to process book sales data in order to first find out
relevant variables and then use these variables to process data so
that the publisher, using them, can estimate sales and from them order
a specific print run, instead of standard ones, which are around 300
copies for standard books and around 5000 for those that are
considered, in advance, {\em best sellers}.

Also, in the next-to-last paragraph, we have laid out our methodology 
and the main contributions of this paper.
\end{verbatim}

~\\
\noindent (2) \emph{sections 4.1 and 4.2 should be moved to section 2. } 

\begin{verbatim}
TODO
\end{verbatim}

~\\
\noindent (3) \emph{In section 4, this manuscript should show a flowchart of the proposed forecasting scheme and describe it. } 

\begin{verbatim}
Taking into account the reviewer's comment, we have improved the methodology description by including a flowchart (see Figure 1 at page 6).
\end{verbatim}

~\\
\noindent (4) \emph{In Section 5, parameter settings of M5P, kNN and RF also should be shown. } 

\begin{verbatim}
TODO
\end{verbatim}


\end{document}
