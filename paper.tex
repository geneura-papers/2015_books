\documentclass{llncs}
\usepackage{graphics}
\usepackage{amssymb}
\usepackage[dvips]{epsfig}
\usepackage[spanish]{babel}
%\usepackage[latin1]{inputenc}

\def\CC{{C\hspace{-.05em}\raisebox{.4ex}{\tiny\bf ++}}~}
\addtolength{\textfloatsep}{-0.5cm}
\addtolength{\intextsep}{-0.5cm}


%%%%%%%%%%%%%%%% Titulo %%%%%%%%%%%%%%%
\title{books prediction}

%%%%%%%%%%%%%%%% autores %%%%%%%%%%%%%%%
\author {
P.A. Castillo et al.
}
\institute{Department of Architecture and Computer Technology. CITIC \\
           University of Granada (Spain) \\
~\\
           e-mail: {\tt pacv@ugr.es}}

\date{} 

\begin{document}
\maketitle

%%%%%%%%%%%%%%%%%%%%%%%%%%%%%%%%%%%%%%%%%%%%%%%%%%%%%%%%%%%%%%
\begin{abstract}

the abstract

% Cuando se lleva a cabo 
% Dado el coste que supone llevar a cabo una tirada editorial, se hace necesario disponer de herramientas autom�ticas, basadas en algoritmos de extracci�n de la informaci�n que ofrezcan estimaciones de la cantidad de copias de cierto libro que se pueden llegar a vender.
% En ese sentido, en colaboraci�n con la empresa PRM Consultores S.C.A. se ha llevado a cabo el an�lisis de un conjunto de datos que describen la tirada editorial en los �ltimos a�os. Como parte del estudio, se han aplicado diversos m�todos de clasificaci�n y predicci�n para determinar cu�ntos libros se pueden llegar a vender.

\end{abstract}

%********************************************************************************
\section{Introduction and State of the Art}



%********************************************************************************
\section{The problem}



%********************************************************************************
\section{Methodology}



%********************************************************************************
\section{Experiments and Results}



%********************************************************************************
\section{Conclusions and Future Work}



%********************************************************************************
\section*{Acknowledgements}
This work has been supported in part by SIPESCA (Programa Operativo FEDER de Andaluc�a 2007-2013), TIN2011-28627-C04-02 and TIN2014-56494-C4-3-P (Spanish Ministry of Economy and Competitivity), SPIP2014-01437 (Direcci{\'o}n General de Tr{\'a}fico), PRY142/14 (Fundaci{\'o}n P{\'u}blica Andaluza Centro de Estudios Andaluces en la IX Convocatoria de Proyectos de Investigaci{\'o}n), and PYR-2014-17 GENIL project (CEI-BIOTIC Granada).


%********************************************************************************
\bibliographystyle{plain}
\bibliography{refs}

\end{document}
